\section{Conclusie}
\label{sec:conclusie}

De hoofdvraag van dit onderzoek luidt:

\begin{quote}
  "\textit{Welke API architectuur is het beste geschikt voor de specifieke eisen van het project, waarbij de communicatie tussen verschillende softwarecomponenten, de ondersteuning van gebruikersinteracties, en de flexibiliteit en onderhoudbaarheid van de code optimaal zijn?}"
\end{quote}

Voor het uiteindelijke project zal er gebruik worden gemaakt van \textbf{REST}
als API architectuur. REST is de meest geschikte keuze voor het project, omdat
het voldoet aan de eisen van het project en het ontwikkelen van de API zonder
externe libraries mogelijk maakt. REST is eenvoudig te implementeren en biedt
een breed scala aan mogelijkheden voor communicatie tussen verschillende
componenten.

Om deze hoofdvraag te beantwoorden, zijn de volgende deelvragen geformuleerd:

\begin{quote}
  "\textit{Welke stappen zijn nodig om de API vanaf de grond af aan op te bouwen?}"
\end{quote}

Uit de ontwikkeling van de API blijkt dat er geen nood is voor specifieke
dataselectie, wat GraphQL juist zo sterk maakt. Ook is er geen nood voor
real-time communicatie of authenticatie.

\begin{quote}
  "\textit{Hoe kunnen REST, SOAP, GraphQL, en gRPC worden vergeleken op basis van de gestelde criteria?}"
\end{quote}

Het project stelt een aantal eisen aan de API-architectuur, waaronder het
ontwikkelen zonder externe libraries. Uit \autoref{sec:resultaten} blijkt dat
\textbf{REST} de meest geschikte keuze is op basis van de gestelde criteria.
