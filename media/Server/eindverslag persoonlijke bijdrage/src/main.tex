\documentclass[10pt]{article}

%/ Use case: margins
\usepackage[letterpaper,
  top=2cm,
  bottom=2cm,
  left=2cm,
  right=2cm,
  marginparwidth=1.75cm]{geometry}
%/ Use case: language and spell checking
\usepackage[utf8]{inputenc}
\usepackage[english]{babel}
%/ Use case: uppercase headers
\usepackage{titlecaps}
\usepackage{sectsty}
%/ Use case: text formatting
\usepackage{url}
\usepackage[outputdir=tmp]{minted}
\usepackage[colorlinks=true, allcolors=teal]{hyperref}
\usepackage{textcomp}
\usepackage{amsmath}
%/ Use case: images
\usepackage{graphicx}
\usepackage[export]{adjustbox}
\usepackage[font=small,labelfont=bf]{caption}
%/ Use case: tables
\usepackage{booktabs}
\usepackage[flushleft]{threeparttable}
\usepackage{pgfplots}
\usepackage{pgfplotstable}

\pgfplotsset{compat=1.18}

\allsectionsfont{\mdseries\scshape}

\title{\titlecap{\scshape Title}}\normalfont
\author{Wessel Tip $<$contact@wessel.gg$>$ (Student number 696770, \url{https://wessel.gg/})}
\date{Computer Engineering at InHolland University of Applied Sciences\linebreak Year 2, Semester 2 (Jan. 2024 - Jun. 2024)}

\begin{document}

\maketitle

\begin{abstract}
In dit verslag zal de ontwikkeling van mijn persoonlijke bijdrage aan het project
Robotica worden besproken. Mijn persoonlijke bijdrage zal de communicatie tussen
de gebruiker, de robot en een aparte dataserver beheren.

Het hele project zal in Csharp worden geschreven door zijn objectgeoriënteerde aard.
\end{abstract}

\tableofcontents

\newpage

\section{Introduction}
\label{sec:introduction}




\bibliographystyle{IEEEtran}
\bibliography{references.bib}

\end{document}
