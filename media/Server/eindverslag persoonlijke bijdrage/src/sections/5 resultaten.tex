\section{Resultaten}
\label{sec:resultaten}

\subsection{Stappen voor het opbouwen van de API vanaf de grond af aan}
Voor de ontwikkeling van de API zijn de volgende stappen doorlopen:
\begin{enumerate}
  \item \textbf{Behoeftenanalyse en Specificatie} --- De eerste stap was het
  identificeren van de eisen voor de API. De volgende eisen zijn opgesteld na
  aanleiding van de behoeftenanalyse:
  \begin{enumerate}
    \item \textbf{Basisfunctionaliteit}
    \begin{itemize}
      \item Ondersteunen van communicatie tussen de gebruiker en de robot.
      \item Mogelijkheid om de robot te besturen en inzicht te krijgen via een
      webinterface.
    \end{itemize}
    \item \textbf{Technische Specificaties}
    \begin{itemize}
      \item De API moet worden opgebouwd zonder het gebruik van externe libraries.
      \item Gebruik van standaard HTTP-methoden (GET, POST, PUT, DELETE).
      \item Het gebruik van JSON voor gegevensoverdracht.
    \end{itemize}
    \item \textbf{Flexibiliteit en Onderhoudbaarheid}
    \begin{itemize}
      \item Eenvoudige structuur die gemakkelijk te onderhouden is.
      \item Flexibel genoeg om toekomstige toevoegingen en aanpassingen te
      ondersteunen zonder dat de hele codebase herschreven moet worden.
    \end{itemize}
    \item \textbf{Gebruikersinterface}
    \begin{itemize}
      \item Weergave van verzamelde data in een overzichtelijke manier.
      \item Presentatie van de mapping en de geplande route van de robot.
      \item Mogelijkheid om diverse commando’s naar de robot te sturen.
    \end{itemize}
  \end{enumerate}
  \item \textbf{Architectuurontwerp} --- Vervolgens zijn de toegestane request types (GET, POST, PUT, DELETE) gedefinieerd, samen met de benoemde routes en hun verwachte responses. Dit heeft geresulteerd in een gedetailleerd ontwerp van de API-structuur.
  \begin{itemize}
    \item \textbf{GET /api/v1/database/metadata}
    \linebreak Geeft basisinformatie over de database terug.
    \item \textbf{GET /api/v1/database}
    \linebreak \textbf{Query Paramters:}
    \begin{itemize}
      \item ?id=$<$string$>$ --- De UUIDv4 van de gewensde map.
      \item ?name=$<$string$>$ --- De naam van de gewensde map.
      \item ?version=$<$int$>$ --- Het versienummer van de gewensde maps
      \item ?all=$<$bool$>$ -- Laat alle maps zien
    \end{itemize}
    \textbf{Return Body:}
    \begin{minted}[linenos,frame=leftline,framesep=3pt]{json}
    [
      {
        "Id": "<string>",
        "Name": "<string>",
        "Version": "<int>",
        "Objects": "<Array<Array<int, int>>"
        "Date": "<DateTime>"
      }, {...}
    ]
    \end{minted}
    Geeft een lijst van alle maps terug, of een specifieke map op basis van de query parameters.
    \item \textbf{POST /api/v1/database}
    \linebreak \textbf{Request Body:}
    \begin{minted}[linenos,frame=leftline,framesep=3pt]{json}
    {
      "name": "<string>",
      "objects": "<Array<Array<int, int>>>"
    }
    \end{minted}
    \textbf{Return Body:}
    \begin{minted}[linenos,frame=leftline,framesep=3pt]{json}
    {
      "message": "<success|fail>",
      "id": "<string>",
    }
    \end{minted}
    Voegt een nieuwe map toe aan de database.
    \item \textbf{DELETE /api/v1/database}
    \linebreak \textbf{Query Paramters:}
    \begin{itemize}
      \item ?id=$<$string$>$ --- De UUIDv4 van de gewensde map.
      \item ?name=$<$string$>$ --- De naam van de gewensde map.
    \end{itemize}
    Verwijdert de desbetreffende map uit de database.
    \item \textbf{GET /api/v1/database/path}
    \linebreak \textbf{Query Paramters:}
    \begin{itemize}
      \item ?id=$<$string$>$ --- De UUIDv4 van de gewensde route.
    \end{itemize}
    \textbf{Return Body:}
    \begin{minted}[linenos,frame=leftline,framesep=3pt]{json}
    {
      "Id": "<string>",
      "Objects": "<Array<Array<int, int>>"
    }
    \end{minted}
    \item \textbf{POST /api/v1/database/path}
    \linebreak \textbf{Request Body:}
    \begin{minted}[linenos,frame=leftline,framesep=3pt]{json}
    {
      "id": "<string>",
      "objects": "<Array<Array<int, int>>>"
    }
    \end{minted}
    \textbf{Return Body:}
    \begin{minted}[linenos,frame=leftline,framesep=3pt]{json}
    {
      "message": "<success|fail>",
      "id": "<string>",
    }
    \end{minted}
    Voegt een nieuwe geplande route toe aan de database.
    \item \textbf{DELETE /api/v1/database/path}
    \linebreak \textbf{Query Paramters:}
    \begin{itemize}
      \item ?id=$<$string$>$ --- De UUIDv4 van de gewensde route.
    \end{itemize}
    Verwijdert de desbetreffende route uit de database.
    \item \textbf{POST /api/v1/database/path/plan}
    \linebreak \textbf{Request Body:}
    \begin{minted}[linenos,frame=leftline,framesep=3pt]{json}
    {
      "id": "<string>",
      "objects": "<Array<Array<int, int>>>"
    }
    \end{minted}
    \textbf{Return Body:}
    \begin{minted}[linenos,frame=leftline,framesep=3pt]{json}
    {
      "message": "<success|fail>",
      "id": "<string>",
    }
    \end{minted}
    Voegt een nieuwe geplande route toe aan de database.
    \item \textbf{POST /api/v1/roomba/control}
    \linebreak \textbf{Request Body:}
    \begin{minted}[linenos,frame=leftline,framesep=3pt]{json}
    {
      "command": "<string>"
    }
    \end{minted}
    Proxies het commando naar de Roomba.
  \end{itemize}
  \item \textbf{UML ontwerpen} --- Een klassendiagram is gemaakt om de structuur
  van de code te visualiseren. Dit diagram is te vinden in \autoref{fig:uml}.

  De UML is opgedeeld in drie delen: de HTTP server, het data parsen en de endpoints.
  De HTTP server is verantwoordelijk voor het ontvangen van de requests en het
  doorsturen naar de juiste endpoint. Het data parsen zorgt ervoor dat de data
  correct wordt verwerkt en opgeslagen. De endpoints zijn de verschillende routes
  die de API aanbiedt en de bijbehorende acties die worden uitgevoerd.
  \item \textbf{Implementatie} --- De API is opgebouwd volgens de specificaties
  van de vorige stappen. Er is gebruik gemaakt van Csharp zonder externe libraries
  om de functionaliteiten van de API te implementeren.
  \item \textbf{Testen} --- Er zijn unit tests ontwikkeld voor alle componenten
  die door de API worden gebruikt. Eventuele problemen die tijdens het testen naar
  voren kwamen zijn verwerkt en opgelost.
  \item \textbf{Onderhoud} --- Na implementatie is de API gemonitord en aangepast
  op basis van feedback. Nieuwe functies zijn toegevoegd en bestaande functies
  zijn verbeterd waar nodig, zonder de bestaande functionaliteit te breken.
\end{enumerate}

\subsection{Vergelijking van REST, SOAP, GraphQL, en gRPC op basis van gestelde criteria}
De API architecturen zijn vergeleken op basis van de volgende criteria:
prestaties, schaalbaarheid, flexibiliteit, onderhoudbaarheid, en geschiktheid
voor de projectseisen.
Hieronder volgt een samenvatting van de resultaten verzameld in
\autoref{sec:theoretisch kader}.

\subsubsection{REST}
\begin{itemize}
    \item \textbf{Voordelen}: REST heeft een eenvoudige structuur, maakt gebruik van standaard HTTP-methoden en JSON, en is gemakkelijk te implementeren. Het is goed schaalbaar door zijn stateless karakter.
    \item \textbf{Nadelen}: REST kan overbodige data teruggeven, is minder geschikt voor complexe query’s, en biedt geen directe ondersteuning voor documentatie en real-time communicatie.
\end{itemize}

\subsubsection{SOAP}
\begin{itemize}
    \item \textbf{Voordelen}: SOAP biedt beveiligingsfuncties en ondersteuning
    voor ACID-transacties. Het maakt gebruik van XML voor berichtuitwisseling en
    ondersteunt verschillende transportprotocollen.
    \item \textbf{Nadelen}: SOAP heeft een ingewikkeld formaat door de
    XML-berichten, wat resulteert in een hogere complexiteit vergeleken met REST
    waar meerdere typfouten op kunnen treden.
\end{itemize}

\subsubsection{GraphQL}
\begin{itemize}
    \item \textbf{Voordelen}: GraphQL biedt efficiëntie door de mogelijkheid om
    specifieke sets data op te vragen, wat over-fetching en under-fetching
    vermindert. Ook biedt GraphQL type controle door middel van schema's.
    \item \textbf{Nadelen}: De serverimplementatie van GraphQL is complexer,
    waardoor het moeilijker kan zijn om zonder externe libraries te implementeren.
\end{itemize}

\subsubsection{gRPC}
\begin{itemize}
    \item \textbf{Voordelen}: gRPC biedt hoge prestaties door het gebruik van
    Protocol Buffers, en is zeer geschikt voor real-time communicatie door
    HTTP/2. Het biedt ook functies zoals load balancing en monitoring.
    \item \textbf{Nadelen}: gRPC is complexer op te zetten en te debuggen, en
    heeft limitaties door het gebruik van HTTP/2, wat niet compatibel is met
    oudere web browsers. Ook heeft gRPC net als GraphQL een complexe serverimplementatie.
\end{itemize}
