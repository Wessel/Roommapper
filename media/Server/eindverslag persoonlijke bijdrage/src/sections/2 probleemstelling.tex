\section{Probleemstelling}
\label{sec: Probleemstelling}

In het project is het belangrijk dat de gebruiker op afstand can communiceren
met de mapping robot. De robot moet de commando's van de gebruiker kunnen
ontvangen en uitvoeren.

De gebruiker moet ook kunnen zien wat de robot allemaal gemapt heeft, om
eventueel in te grijpen als dit niet klopt.

De robot vereist een structuele manier om deze gegevens uit te wisselen, en
de optie om meerdere robots tegelijkertijd te besturen. De aandachtspunten
hiervoor zijn:

\subsection{Toegankelijkheid van de diverse functies}
Het interface dat de gebruiker ziet moet alle verzamelde data laten zien.
Hij zal de mapping en de geplande route moeten presenteren in een overzichtelijke
manier. Ook moet de robot diverse commando's vanuit de gebruiker kunnen ontvangen
om zo de gewenste taken uit te kunnen voeren.

\subsection{Schaalbaarheid en Prestaties}
Naarmate het project groeit, moeten de communicatie kanalen tussen de verschillende
componenten schaalbaar blijven zonder dat dit ten koste gaat van de prestaties.
Dit betekent dat de gekozen API-oplossing in staat moet zijn om een toenemend aantal
verzoeken af te handelen zonder vertragingen of systeemuitval.

\subsection{Onderhoudbaarheid en Uitbreidbaarheid}
De gekozen API-architectuur moet eenvoudig te onderhouden zijn, met een heldere
en overzichtelijke structuur. De architectuur moet ook felxiebel genoeg zijn
om toevoegingen en aanpassingen te ondersteunen zonder dat het codebase herschreven
moet worden.
