\section{Probleemstelling}
\label{sec: Probleemstelling}

\subsection{Probleemanalyse}
\label{ssec:probleemanalyse}

API's (\textit{Application Programming Interfaces}) zijn onmisbare componenten in moderne
software. Door middel van een API in combinatie met een set van afspraken
(een \textit{protocol}) is het mogelijk meerdere softwarecomponenten met elkaar
te integreren. Deze API's maken de communicatie tussen verschillende
softwarecomponenten mogelijk.\cite{Souza_2012}

Er zijn over de jaren vele verschillende manieren bedacht voor het ontwerpen
en implementeren van API's, elk met zijn eigen voordelen en nadelen.
\cite{Śliwa_Pańczyk_2021} In dit onderzoek wordt er een vergelijking gesteld
tussen vier verschillende API architecturen (REST, SOAP, GraphQL en gRPC)
om te bepalen welke het meest geschikt is voor het project.

Bij het kiezen van een API architectuur wordt er rekening gehouden met een bepaald
aantal criteria. De API moet ontwikkelt worden zonder dat er externe libraries
gebruikt worden. Dit betekent dat de API vanaf de grond af aan moet worden opgebouwd.

De API moet ook in staat zijn om de communicatie tussen de verschillende componenten
van de robot te ondersteunen. Dit betekent dat de API de communicatie tussen de
gebruiker en de robot moet ondersteunen, en de communicatie tussen de verschillende
robots onderling.

De interface die wordt gepresenteerd aan de eindgebruiker dient alle verzamelde
data te laten zien. Hij zal de mapping en de geplande route moeten presenteren
in een overzichtelijke manier. Ook moet de robot diverse commando's vanuit de
gebruiker kunnen ontvangen om zo de gewenste taken uit te kunnen voeren.

De gekozen API-architectuur moet eenvoudig te onderhouden zijn, met een heldere
en overzichtelijke structuur. De architectuur moet ook felxiebel genoeg zijn
om toevoegingen en aanpassingen te ondersteunen zonder dat de codebase herschreven
moet worden.

De robot vereist een structuele manier om deze gegevens uit te wisselen, en
de optie om meerdere robots tegelijkertijd te besturen.

Een bijkomende eis vanuit de opdrachtgever is dat de software van het beroepsproduct
object-georieënteerd is.

\subsection{Vraagstelling}
De hoofdvraag van dit onderzoek luidt:

\begin{quote}
  "\textit{Welke API architectuur is het beste geschikt voor de specifieke eisen
  van het project, waarbij de communicatie tussen verschillende
  softwarecomponenten, de ondersteuning van gebruikersinteracties, en de
  flexibiliteit en onderhoudbaarheid van de code optimaal zijn?}"
\end{quote}

Om deze hoofdvraag te beantwoorden worden de volgende deelvragen geformuleerd:

Voorafgaand aan het project zijn in \autoref{ssec:probleemanalyse} een aantal
eisen opgesteld, zoals dat de API moet worden opgebouwd zonder het benut van
externe libraries. De volgende deelvraag zal dit beantwoorden:

\begin{quote}
  "\textit{Welke stappen zijn nodig om de API vanaf de grond af aan op te bouwen?}"
\end{quote}

Ook zal er gekeken worden welk architectuur het beste bij het project past. Om
dit te beantwoorden is de volgende deelvraag geformuleerd:

\begin{quote}
  "\textit{Hoe kunnen REST, SOAP, GraphQL, en gRPC worden vergeleken op basis
  van de gestelde criteria?}"
\end{quote}
