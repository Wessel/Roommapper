Er zijn vele opties om data te verzenden tussen client en server.
Voor dit project is er voor gekozen om data te serialiseren naar JSON.
Dit is een veelgebruikte methode om data te verzenden tussen client en server.
Het voordeel van JSON is dat het een lichtgewicht formaat is en dat het makkelijk te lezen is.
Dit is handig voor debugging en het is makkelijk om te zetten naar een object in JavaScript.
Een nadeel van JSON is dat het niet binair is en dat het niet zo snel is als binair.
Dit is echter geen probleem voor dit project, omdat de hoeveelheid data die verstuurd wordt klein is.
\subsection{Serialisatie}
\label{sec:serialisatie}
% Path: 2 data serialisatie.tex

