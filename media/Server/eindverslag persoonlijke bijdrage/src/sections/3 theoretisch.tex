\section{theoretisch Kader}
\label{sec:theoretisch kader}

\subsection{API-Architecturen}
API's (Application Programming Interfaces) zijn onmisbare componenten in moderne
software. Deze API's maken de communicatie tussen verschillende
softwarecomponenten mogelijk.\cite{Souza_2012}

Er zijn over de jaren vele verschillende manieren bedacht voor het ontwerpen
en implementeren van API's, elk met zijn eigen voordelen en nadelen.
\cite{Śliwa_Pańczyk_2021}

Dit hoofdstuk zal vier van de meest gebruikte API architecturen bespreken.

\subsection{RESTful (Representational State Transfer)}
\label{ssec:rest}
REST is een architectuur dat wordt gebruikt voor het communiceren van gegevens
om netwerkapplicaties te ontwerpen.

Het maakt gebruik van standaard HTTP-methoden (zoals GET, POST, PUT, DELETE)
om gegevens te communiceren tussen clients en servers.\cite{masse2011}

Door de simpiliciteit en het al gebruik maken van de bestaande HTTP protocollen
is REST een populaire keuze voor API's. Dit maakt het ook erg geschikt voor
web gebaseerde applicaties.

\subsubsection{Voordelen van REST}
\label{sssec:voordelen en nadelen van rest}
\begin{enumerate}
  \item \textbf{Eenvoudige structuur} --- gebaseerd op standaard HTTP-methoden
   en JSON
  \item \textbf{Makkelijke implementatie} --- meeste talen ondersteunen de basis
   van webrequests en JSON serialisatie al
  \item \textbf{Schaalbaarheid} --- RESTful API's zijn stateless en kunnen
   eenvoudig horizontaal geschaald worden, echter zal de prestatie wel minder
   zijn dan andere methodes zoals gRPC of GraphQL.\cite{Śliwa_Pańczyk_2021}
\end{enumerate}

\subsubsection{Nadelen van REST}
Echter heeft REST ook zijn nadelen, sommige hiervan zijn:
\begin{enumerate}
  \item \textbf{Overbodig veel data} --- Minder geschikt voor complexere query's
   en datastructuren (over-fetching en under-fetching)
  \item \textbf{Documentatie} --- Geen directe ondersteuning voor documentatie,
   schemas en typecontrole
  \item \textbf{Real-time} --- Doordat de connectie niet levend wordt gehouden nadat
   de aanvrag is afgehandeld is het minder geschikt voor real-time communicatie.
\end{enumerate}

\subsection{SOAP (Simple Object Access Protocol)}
SOAP is een protocol voor uitwisseling van informatie netzoals REST vermeld in
\autoref{ssec:rest}. Echter is SOAP meer complex en minder populair dan REST.

SOAP maakt gebruik van XML voor het sturen van berichten en ondersteunt
in tegenstelling tot REST verschillende transportprotocollen zoals
HTTP en SMTP. SOAP biedt ook robuuste beveiligings- en transactiebeheerfuncties
wat REST niet zo maar heeft.\cite{Śliwa_Pańczyk_2021,w3c}

\subsubsection{Voordelen van SOAP}
\begin{enumerate}
    \item \textbf{Beveiliging} --- Sterke beveiligingsfuncties met behulp van
    WS-Security en ondersteuning voor ACID transacties.
\end{enumerate}

\subsubsection{Nadelen van SOAP}
Net zoals REST heeft SOAP ook zijn nadelen:
\begin{enumerate}
    \item \textbf{Ingewikkeld formaat} --- Complexiteit en overhead door XML-berichten
    \item \textbf{Prestatieproblemen} --- Langzamere prestaties vergeleken met REST
\end{enumerate}

\subsection{GraphQL}
GraphQL is een querytaal voor API's die is ontwikkeld door Facebook in 2012\cite{facebook}.
In tegenstelling tot REST en SOAP, waarbij de server bepaalt welke gegevens
worden geretourneerd, stelt GraphQL clients in staat om een specifieke set data
op te vragen. Ook biedt GraphQL een sterke typecontrole en introspectie aan door
middel van schemas.\cite{Hartig}

\subsubsection{Voordelen van GraphQL}
Voordelen van GraphQL:
\begin{enumerate}
    \item \textbf{Efficiëntie} --- Door de query aard van GraphQL is het
     makkelijk om snel en flexibele bepaalde stukken data aan te vragen.
    \item \textbf{Gegevensbesparing} --- Omdat de gebruiker alleen maar de
     gegevens ontvangt die zij nodig hebben, vermindert dit over-fetching en
     under-fetching waardoor er meer bandbreedte en rekenkracht wordt bespaard.\cite{Hartig}
    \item \textbf{Documentatie} --- Sterke typecontrole en introspectie door het
     gebruik van schemas.
     \item \textbf{compatibiliteit} --- Ook al is het niet gewenst, GraphQL
      is compatibiel met RESTful clients, hierdoor kan de implementatie vrij
      simpel zijn als de kracht van GraphQL niet nodig is.
\end{enumerate}

\subsubsection{Nadelen van GraphQL}
Nadelen van GraphQL:
\begin{enumerate}
    \item \textbf{Intergratie} --- Door de aard van GraphQL is de
     serverimplementatie veel complexer, en daarom ook aangeraden om een library
     te gebruiken.
    \item \textbf{Complexiteit} --- Mogelijkheid van te complexe queries die
     de serverbelasting verhogen
\end{enumerate}

\subsection{gRPC (Google Remote Procedure Call)}
gRPC is een modern RPC-framework dat in 2016 door Google is ontwikkeld. Het maakt
gebruik van HTTP/2 voor transport, Protocol Buffers voor berichtserialisatie en
biedt functies zoals load balancing en monitoring. Door zijn load balancing
functies is gRPC zeer geschikt voor high-performance en real-time communicatie.\cite{google,Śliwa_Pańczyk_2021}

Echter is het wel de moeilijkste methoden om te implementeren en te onderhouden.

\subsubsection{Voordelen van gRPC}
\begin{enumerate}
    \item \textbf{Prestatie} --- Hoge prestaties door het gebruik van protocol
     buffers, berichten zijn tot 30\% kleiner dan JSON.
    \item \textbf{Real-time} --- In tegenstelling tot REST, GraphQL en SOAP
     is gRPC wel geschikt voor real-time communicatie
\end{enumerate}

\subsubsection{Nadelen van gRPC}
\begin{enumerate}
    \item \textbf{Complexiteit} --- Complexer dan alle andere methodes vermeld
     om op te zetten en te debuggen
     \item \textbf{Limitaties} --- Door het vele gebruik van HTTP/2.0 is het niet
      compatibel met oudere web browsers. Ook is er veel minder bekend over gRPC
      dan de andere methodes door zijn relatief nieuwe staat.
\end{enumerate}
